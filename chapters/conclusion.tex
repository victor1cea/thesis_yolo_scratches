\section{Conclusions?}
Now it's time to draw the line and put everything together. The scope of this section is to highlight important discoveries, results and limitations. \\
The first relevant discovery was understanding the dataset on hand. The dataset contained thick edges that looked like scratches and artifacts from previous layer, so that during the annotation process the current bitmask and previous bitmask had to be used. This lead to the idea of the bitmask integration via the two free color channels and the model became very robust thanks to the additional context information. \\
The first models were very sensitive and the slightest hint of color degradation triggered a false detection. The problem was that the scratches were annotated very inconsistently. Therefore, the prominence metric was developed and it help providing a consistent labeling defined by a clear threshold.\\
Despite this great metric, some images contained nice examples of prominent scratches, but also faded scratches. Now the problem was how to properly filter out those edge cases during development. Here came the windowing process to the rescue, which gave a granular control over the scratch selection of an image by including only the desired windows with prominent scratches. Later on, the lowering of the prominence threshold was possible due to the increased robustness and stability of the model and the model had access to a bit more data. This was only one initial and crucial benefit of windowing. \\
The square shape of the input images avoided unnecessary image padding and toggled the data shuffling on , which removed any concerns regarding the padding methods and made the model overfit less. Also, the smaller windows made the possible to increase the batch size, so that the model can have a more stable batch normalization. \\
The next benefit provided by the windowing technique is the better recall. Because the neighboring windows overlap and the also extra redunant windows
