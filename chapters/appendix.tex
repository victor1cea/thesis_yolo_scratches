
\begin{figure}[!h]
\centering
\begin{subfigure}{.8\textwidth}
  \centering
  \includegraphics[width=\linewidth]{images/implementation/results/bm/layer_00093}
  \caption{Layer \#93 contains}
\end{subfigure}

\begin{subfigure}{.8\textwidth}
  \centering
  \includegraphics[width=\linewidth]{images/implementation/results/bm/bitmask_00093}
  \caption{Bitmask of layer \#93}
\end{subfigure}

\begin{subfigure}{.8\textwidth}
  \centering
  \includegraphics[width=\linewidth]{images/implementation/results/bm/bitmask_00092}
  \caption{Bitmask of previous layer \#92}
\end{subfigure}

\caption{The layer 93 contains closely positioned edges and artifacts from previous. Therefore, this is a good example to highlight different bitmask integration methods. }
\label{impl:layer_example}
\end{figure}

\begin{figure}[!h]
\centering

\begin{subfigure}{.9\textwidth}
  \centering
  \includegraphics[width=\linewidth]{images/implementation/results/bm/bm_sobel}
  \caption{Using a bitmask and it's sobel filtered version}
\end{subfigure}

\begin{subfigure}{.9\textwidth}
  \centering
  \includegraphics[width=\linewidth]{images/implementation/results/bm/and}
  \caption{Using current and a logical and between the image and the bitmask.}
\end{subfigure}

\caption{More alternative bitmask integration methods.}
\label{app:bm_compare_ext}
\end{figure}
